\section{Introduction}
\label{SEC:introduction}

\cappos{I think there is some quote about it being hard to keep secrets.}

Secrets are hard to keep, especially when they are shared.  Computer systems
are not immmune to this phenomenon.  
One of the most challenging problems
involves the effective provisioning of secrets.  Common secrets of this
sort are TLS keys and certificates, passwords, internal server names, and SSH 
keys \cappos{possible cites}.  

Unfortunately, provisioning secrets securely is hard, leading to unintended
disclosure of sensitive information.  Fairly commonly, secrets are 
accidentally checked into version control systems, emitted in logs, or 
...   Furthermore, due to the difficulty of secure provisioning, secrets
are also commonly shared between test and production environments, leading
to both accidents and malicious behavior.
\cappos{mention you can give Travis secrets in some way.  However, it leaks 
those secrets by printing them in log files (unencrypted).}

\cappos{Perhaps motivate how containers / service computing is increasing 
the number of small, distributed items in a server here.}

The reason why it is hard to securely provision
secrets is due to four factors.  
\begin{itemize}
\item {\bf API.} First, the mechanisms used to provide
secrets, such as environment variables or custom APIs, require substantial
vigilence from dev ops personnel to avoid being leaked~\cappos{cite 
Perforce was storing FreeBSD core
dumps.  Those core dumps had environment variables (w/ secrets) that were
available publicly accidentally.}   
\item {\bf persistance.} Second, secrets must be stored
\emph{somewhere} to survive power cycling.  There needs to be a way to 
securely persist secrets so that an attacker cannot gain access.
\item {\bf distribution.} Third, the secrets must be securely distributed
to an array of different container or service instances.  Ensuring that
each recipient is authorized to receive a secret is paramount, as is
following the principle of least privilege to provide only the needed
secrets.
\item {\bf rotation.} Finally, despite the best effort of systems
implementers, history has shown that security bugs arise in even
the best tested code.  It is important to have a secure, responsive, and
usable mechanism for revoking and rotating secrets.  
\end{itemize}


%Brief:
%dm-verify / secure dep / image resolution.  Just want to be sure we have 
%some way to explain how we know who to give the secret to in the first place
%and what information to share.
%Run system services in containers
%
%Core: 
%+ Secure container platform / swarm setup (Swarmkit / Notary)
%+ Root of trust rotation
%+ Rotate keys / certs in and out of the hands of third parties
%+ secure node introduction.  How infrakit gets keys there in the first place.
%(Infrakit / Swarmkit)
%+ reverse uptime to bound liveness of nodes.  (Infrakit)
%(Can you enforce this on a compromised system?)  
%+ Mutual TLS between nodes. (Is this Swarmkit???)




In this paper, we present \sysname, a novel technique for addressing
the secret distribution problem in distributed systems.  While
\sysname is part of a broader ecosystem, it primarily extends three key 
components (SwarmKit, InfraKit, and Notary) to distribute, persist, and 
rotate secrets.  Containers are securely validated for both correctness and
freshness.  Validation is done prior to secret information being shared with 
them through the use of cryptographic node
identities.  Much like the rights field and check bits concepts common
to many capability-based access control systems, these identities embed
node roles, root CA information, and proof of the right to join the swarm
in X509 certificates.  

Shared secrets are also used to transparently set up secure intra-service 
and inter-service communications channels.  All communication between
services \cappos{even on the same physical server?} is done using mutual
TLS authentication.  This prevents man-in-the-middle attackers from
intercepting communications between services or impersonating a service
in communications.  

\sysname is designed to minimize the impact of a successful attack, with
key rotation a central part of the design.  All keys can be transparently
rotated without operator or service intervention \cappos{???}.  
This includes rotating the root of trust inside of a set of containers,
possibly migrating trust to and from third party CAs as desired.   The
ability to quickly and seamlessly rotate keys  means that keys may be 
replaced often --- limiting an attacker's ability to cause damage after 
a successful breach.  In fact, in production many users of \sysname
use this mechanism to automatically rotate their secrets every hour.


To ensure that images are correct and fresh, this work introduces the 
concept of \emph{reverse uptime}.  Reverse uptime provides a duration which
a service is permitted to live.  Once this value reaches zero, the service
is automatically killed \cappos{How does this work?  Is the cryptographic
node identity also revoked?  As I understand it, these would be separate
mechanisms...}  This ensures that services constantly are installing fresh
images.  It also may have the side effect of reducing the amount of time an
attaker that compromises a service has to utilize that access.

Since \sysname is designed to be seamless to deploy, it is easy use 
in many 
\sysname has been deployed in production in Docker since 1.13, leading to 
use in millions of swarms.  This practical experience, uncovered several
non-obvious lessons from \sysname.  First, while \sysname does not have the 
same security guarantees as specialized designs (e.g., SGX, which requires
specialized hardware), the usability and seamless nature of \sysname makes it
possible to quickly deploy in many environments.    Second, ..., Finally, ...
As a result, \sysname has improved the security of millions of containers.


The primary contributions of this work are as follows:

\begin{itemize}

\item This work describes a series of novel techniques to enable secrets to
be securely shared between containers.  This includes techniques to
securely introduce nodes to each other, automatic provisioning of mutual
TLS connections (even for the application's connections), as well as reverse 
uptime and rotation of secrets to bound the impact of a compromise.

\item \sysname was designed based upon pragmatic decisions that maximize 
ease of deployment.  For example, \sysname maps in each secret
as though it is a file in the file system.  This means that applications
need not be modified to use \sysname.  

\item Careful engineering of \sysname demonstrates that security need not
come at the cost of performance.  \sysname adds only X\% overhead in 
bandwidth, a XXms container startup latency, and XXXX when rotating
secrets.


\end{itemize}



